\documentclass{article}
\usepackage[utf8]{inputenc}
\usepackage[norsk, english]{babel}
\usepackage{cite}
\usepackage{color}
\usepackage{rotating}
\usepackage{adjustbox}

\newcommand{\HRule}{\rule{\linewidth}{0.5mm}}
\setcounter{tocdepth}{5}

\begin{document}
\title{Strengths and Weaknesses of the Agile Retropsective}

\author{Alf Magnus Stålesen, Bjørn Dølvik}
\begin{titlepage}
\begin{center}

\textsc{\LARGE Norwegian University of Science and Technology}\\[1.5cm]

\textsc{\Large }\\[0.5cm]

% Title
\HRule \\[0.4cm]
{ \huge \bfseries Strengths and Weaknesses of the Agile Retropsective\\[0.4cm] }

\HRule \\[1.5cm]

{\large Authors:}\\

Alf \textsc{Magnus Stålesen}\\
Bjørn \textsc{Dølvik}\\[1.0cm]

{\large Supervisor:}\\

Torgeir \textsc{Dingsøyr}\\[1.0cm]

{\large \today}

\end{center}
\end{titlepage}

\begin{abstract}
\end{abstract}
\clearpage

\tableofcontents
\clearpage

\part{Introduction}
\section{Thesis}
Over time the agile retrospective has a tendency to become more repeating and less engaging for the participants in an development project. This can result in the retrospective loosing its value as a place for improvement. 

This article is going to review the agile retrospective with the aim of increasing knowledge sharing within a company. The goal is to review the practices used in todays working enviroment by finding the strengths and weaknesses. Thus give a set of guidelines that can be used for developing a tool that can support the retrospective in not becomming too repetetive and loosing its value.
\clearpage

\part{Method}
\clearpage

\part{Results}
\clearpage

\part{Discussion}
\clearpage

\bibliography{library}
\bibliographystyle{plain}
\end{document}